\hypertarget{lista_8cpp}{\section{Dokumentacja pliku /home/karolina/\-Pulpit/\-L\-A\-B3nowe/prj/lista.cpp}
\label{lista_8cpp}\index{/home/karolina/\-Pulpit/\-L\-A\-B3nowe/prj/lista.\-cpp@{/home/karolina/\-Pulpit/\-L\-A\-B3nowe/prj/lista.\-cpp}}
}


Definicja konstruktora \hyperlink{class_list}{List}.  


{\ttfamily \#include \char`\"{}lista.\-hpp\char`\"{}}\\*
Wykres zależności załączania dla lista.\-cpp\-:


\subsection{Opis szczegółowy}
Definicja konstruktora \hyperlink{class_list}{List}. Przeciazenie operatora$\ast$.

Przeciazenie operatora ==.

Definicja konstruktora Show\-\_\-\-List.

Definicja konstruktora Pop\-\_\-\-Back.

Definicja konstruktora Pop\-\_\-\-Front.

Definicja konstruktora Push\-\_\-\-Front.

Definicja konstruktora Size.

Definicja konstruktora Isempty.

Definicja destruktora \hyperlink{class_list}{List}.

Prztwarza obiekt do znaczniku front i back, przypisuje N\-U\-L\-L i liczy ile el. ma lisya

Usuwa wszystkie elementy

Sprawdza czy lista jest pusta

Rozmiar listy

Dodawanie na poczatek

Jest to dla stosu. Dodaje na poczatek.

Jest to dla kolejki. Dodaje na koniec.

Pokazuje liste.

sprawdzanie czy nie sa puste, jezeli nie to\-: 1-\/ 1$>$2 wykonywany 1 if 2$>$1 drugi if else kiedy takie same

Przeciazenie operatora$\ast$ 

Definicja w pliku \hyperlink{lista_8cpp_source}{lista.\-cpp}.

