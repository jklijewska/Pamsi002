\documentclass{book}
\usepackage[a4paper,top=2.5cm,bottom=2.5cm,left=2.5cm,right=2.5cm]{geometry}
\usepackage{makeidx}
\usepackage{natbib}
\usepackage{graphicx}
\usepackage{multicol}
\usepackage{float}
\usepackage{listings}
\usepackage{color}
\usepackage{ifthen}
\usepackage[table]{xcolor}
\usepackage{textcomp}
\usepackage{alltt}
\usepackage{ifpdf}
\ifpdf
\usepackage[pdftex,
            pagebackref=true,
            colorlinks=true,
            linkcolor=blue,
            unicode
           ]{hyperref}
\else
\usepackage[ps2pdf,
            pagebackref=true,
            colorlinks=true,
            linkcolor=blue,
            unicode
           ]{hyperref}
\usepackage{pspicture}
\fi
\usepackage[utf8]{inputenc}
\usepackage{polski}
\usepackage[T1]{fontenc}

\usepackage{mathptmx}
\usepackage[scaled=.90]{helvet}
\usepackage{courier}
\usepackage{sectsty}
\usepackage{amssymb}
\usepackage[titles]{tocloft}
\usepackage{doxygen}
\lstset{language=C++,inputencoding=utf8,basicstyle=\footnotesize,breaklines=true,breakatwhitespace=true,tabsize=4,numbers=left }
\makeindex
\setcounter{tocdepth}{3}
\renewcommand{\footrulewidth}{0.4pt}
\renewcommand{\familydefault}{\sfdefault}
\hfuzz=15pt
\setlength{\emergencystretch}{15pt}
\hbadness=750
\tolerance=750
\begin{document}
\hypersetup{pageanchor=false,citecolor=blue}
\begin{titlepage}
\vspace*{7cm}
\begin{center}
{\Large My Project }\\
\vspace*{1cm}
{\large Wygenerowano przez Doxygen 1.8.3.1}\\
\vspace*{0.5cm}
{\small N, 16 mar 2014 22:04:12}\\
\end{center}
\end{titlepage}
\clearemptydoublepage
\pagenumbering{roman}
\tableofcontents
\clearemptydoublepage
\pagenumbering{arabic}
\hypersetup{pageanchor=true,citecolor=blue}
\chapter{Dokumentacja zadania P\-A\-M\-S\-I L\-A\-B 3}
\label{index}\hypertarget{index}{}\begin{DoxyAuthor}{Autor}
Justyna Klijewska 
\end{DoxyAuthor}
\begin{DoxyDate}{Data}
16.\-03.\-2014 
\end{DoxyDate}
\begin{DoxyVersion}{Wersja}
0.\-1 
\end{DoxyVersion}

\chapter{Indeks klas}
\section{Lista klas}
Tutaj znajdują się klasy, struktury, unie i interfejsy wraz z ich krótkimi opisami\-:\begin{DoxyCompactList}
\item\contentsline{section}{\hyperlink{class_list}{List} }{\pageref{class_list}}{}
\item\contentsline{section}{\hyperlink{struct_list_ele}{List\-Ele} }{\pageref{struct_list_ele}}{}
\item\contentsline{section}{\hyperlink{class_queue}{Queue} }{\pageref{class_queue}}{}
\item\contentsline{section}{\hyperlink{class_stos}{Stos} }{\pageref{class_stos}}{}
\item\contentsline{section}{\hyperlink{class_tablica}{Tablica} }{\pageref{class_tablica}}{}
\item\contentsline{section}{\hyperlink{class_zegar}{Zegar} }{\pageref{class_zegar}}{}
\end{DoxyCompactList}

\chapter{Indeks plików}
\section{Lista plików}
Tutaj znajduje się lista wszystkich plików z ich krótkimi opisami\-:\begin{DoxyCompactList}
\item\contentsline{section}{/home/karolina/\-Pulpit/\-L\-A\-B3nowe/prj/\hyperlink{kolejka_8hpp}{kolejka.\-hpp} \\*Definicja klasy \hyperlink{class_queue}{Queue} }{\pageref{kolejka_8hpp}}{}
\item\contentsline{section}{/home/karolina/\-Pulpit/\-L\-A\-B3nowe/prj/\hyperlink{lista_8cpp}{lista.\-cpp} \\*Definicja konstruktora \hyperlink{class_list}{List} }{\pageref{lista_8cpp}}{}
\item\contentsline{section}{/home/karolina/\-Pulpit/\-L\-A\-B3nowe/prj/\hyperlink{lista_8hpp}{lista.\-hpp} \\*Definicja Struktury \hyperlink{struct_list_ele}{List\-Ele} }{\pageref{lista_8hpp}}{}
\item\contentsline{section}{/home/karolina/\-Pulpit/\-L\-A\-B3nowe/prj/\hyperlink{main_8cpp}{main.\-cpp} }{\pageref{main_8cpp}}{}
\item\contentsline{section}{/home/karolina/\-Pulpit/\-L\-A\-B3nowe/prj/\hyperlink{plik_8cpp}{plik.\-cpp} \\*Definicja funkcji Read }{\pageref{plik_8cpp}}{}
\item\contentsline{section}{/home/karolina/\-Pulpit/\-L\-A\-B3nowe/prj/\hyperlink{plik_8hpp}{plik.\-hpp} \\*Definicja funkcji Read }{\pageref{plik_8hpp}}{}
\item\contentsline{section}{/home/karolina/\-Pulpit/\-L\-A\-B3nowe/prj/\hyperlink{stos_8hpp}{stos.\-hpp} \\*Definicja klasy \hyperlink{class_stos}{Stos} }{\pageref{stos_8hpp}}{}
\item\contentsline{section}{/home/karolina/\-Pulpit/\-L\-A\-B3nowe/prj/\hyperlink{tablica_8cpp}{tablica.\-cpp} \\*Definicja konstruktora Pusch }{\pageref{tablica_8cpp}}{}
\item\contentsline{section}{/home/karolina/\-Pulpit/\-L\-A\-B3nowe/prj/\hyperlink{tablica_8hpp}{tablica.\-hpp} \\*Definicja klasy \hyperlink{class_tablica}{Tablica} }{\pageref{tablica_8hpp}}{}
\item\contentsline{section}{/home/karolina/\-Pulpit/\-L\-A\-B3nowe/prj/\hyperlink{zegar_8cpp}{zegar.\-cpp} \\*Definicja metody Start }{\pageref{zegar_8cpp}}{}
\item\contentsline{section}{/home/karolina/\-Pulpit/\-L\-A\-B3nowe/prj/\hyperlink{zegar_8hpp}{zegar.\-hpp} \\*Definicja klasy \hyperlink{class_tablica}{Tablica} }{\pageref{zegar_8hpp}}{}
\end{DoxyCompactList}

\chapter{Dokumentacja klas}
\hypertarget{class_list}{\section{Dokumentacja klasy List}
\label{class_list}\index{List@{List}}
}


{\ttfamily \#include $<$lista.\-hpp$>$}



Diagram współpracy dla List\-:
\subsection*{Metody publiczne}
\begin{DoxyCompactItemize}
\item 
\hyperlink{class_list_a64d878a92d11f7c63c70cbe4e7dd4176}{List} ()
\item 
\hyperlink{class_list_a70aecf37bd9d779a394e4d50377fbf5f}{$\sim$\-List} ()
\item 
void \hyperlink{class_list_ae14e825ab502fe31686bf3059ed85ed0}{Isempty} ()
\item 
int \hyperlink{class_list_a00e0054a58302c9eceb94d2ca884e6c5}{Size} ()
\item 
void \hyperlink{class_list_a31fbd443a2454901d82e4baa1732fe62}{Push\-\_\-\-Front} (double number)
\item 
double \hyperlink{class_list_a60d28fbb02bd3fc770ba0627d9345dde}{Pop\-\_\-\-Front} ()
\item 
double \hyperlink{class_list_a8b06ea3ceef6bb1b261656e78e1ba6e7}{Pop\-\_\-\-Back} ()
\item 
void \hyperlink{class_list_a25ab387de5733d3a908b730877b0f260}{Show\-\_\-\-List} ()
\item 
\hyperlink{class_list}{List} \& \hyperlink{class_list_a7c478a92a9c02c8948e4495ab8e9acc1}{operator==} (\hyperlink{class_list}{List} \&lista)
\item 
\hyperlink{class_list}{List} \& \hyperlink{class_list_ae49de6522570c22c3dc2d695cbb4ecbf}{operator$\ast$} (double M)
\end{DoxyCompactItemize}
\subsection*{Atrybuty prywatne}
\begin{DoxyCompactItemize}
\item 
\hyperlink{struct_list_ele}{List\-Ele} $\ast$ \hyperlink{class_list_a1bca2e220bb62fc97b3d2952ee6aadec}{front}
\item 
\hyperlink{struct_list_ele}{List\-Ele} $\ast$ \hyperlink{class_list_ab64659e8fecbb40c66a0f63e6b6e14cb}{back}
\item 
unsigned \hyperlink{class_list_aa9b0031d0c9e9e4805c021fbbcc41b14}{counter}
\end{DoxyCompactItemize}


\subsection{Opis szczegółowy}


Definicja w linii 21 pliku lista.\-hpp.



\subsection{Dokumentacja konstruktora i destruktora}
\hypertarget{class_list_a64d878a92d11f7c63c70cbe4e7dd4176}{\index{List@{List}!List@{List}}
\index{List@{List}!List@{List}}
\subsubsection[{List}]{\setlength{\rightskip}{0pt plus 5cm}List\-::\-List (
\begin{DoxyParamCaption}
{}
\end{DoxyParamCaption}
)}}\label{class_list_a64d878a92d11f7c63c70cbe4e7dd4176}


Definicja w linii 9 pliku lista.\-cpp.

\hypertarget{class_list_a70aecf37bd9d779a394e4d50377fbf5f}{\index{List@{List}!$\sim$\-List@{$\sim$\-List}}
\index{$\sim$\-List@{$\sim$\-List}!List@{List}}
\subsubsection[{$\sim$\-List}]{\setlength{\rightskip}{0pt plus 5cm}List\-::$\sim$\-List (
\begin{DoxyParamCaption}
{}
\end{DoxyParamCaption}
)}}\label{class_list_a70aecf37bd9d779a394e4d50377fbf5f}


Definicja w linii 20 pliku lista.\-cpp.



\subsection{Dokumentacja funkcji składowych}
\hypertarget{class_list_ae14e825ab502fe31686bf3059ed85ed0}{\index{List@{List}!Isempty@{Isempty}}
\index{Isempty@{Isempty}!List@{List}}
\subsubsection[{Isempty}]{\setlength{\rightskip}{0pt plus 5cm}void List\-::\-Isempty (
\begin{DoxyParamCaption}
{}
\end{DoxyParamCaption}
)}}\label{class_list_ae14e825ab502fe31686bf3059ed85ed0}


Definicja w linii 36 pliku lista.\-cpp.



Oto graf wywoływań tej funkcji\-:


\hypertarget{class_list_ae49de6522570c22c3dc2d695cbb4ecbf}{\index{List@{List}!operator$\ast$@{operator$\ast$}}
\index{operator$\ast$@{operator$\ast$}!List@{List}}
\subsubsection[{operator$\ast$}]{\setlength{\rightskip}{0pt plus 5cm}{\bf List} \& List\-::operator$\ast$ (
\begin{DoxyParamCaption}
\item[{double}]{M}
\end{DoxyParamCaption}
)}}\label{class_list_ae49de6522570c22c3dc2d695cbb4ecbf}


Definicja w linii 222 pliku lista.\-cpp.

\hypertarget{class_list_a7c478a92a9c02c8948e4495ab8e9acc1}{\index{List@{List}!operator==@{operator==}}
\index{operator==@{operator==}!List@{List}}
\subsubsection[{operator==}]{\setlength{\rightskip}{0pt plus 5cm}{\bf List} \& List\-::operator== (
\begin{DoxyParamCaption}
\item[{{\bf List} \&}]{lista}
\end{DoxyParamCaption}
)}}\label{class_list_a7c478a92a9c02c8948e4495ab8e9acc1}


Definicja w linii 152 pliku lista.\-cpp.



Oto graf wywołań dla tej funkcji\-:


\hypertarget{class_list_a8b06ea3ceef6bb1b261656e78e1ba6e7}{\index{List@{List}!Pop\-\_\-\-Back@{Pop\-\_\-\-Back}}
\index{Pop\-\_\-\-Back@{Pop\-\_\-\-Back}!List@{List}}
\subsubsection[{Pop\-\_\-\-Back}]{\setlength{\rightskip}{0pt plus 5cm}double List\-::\-Pop\-\_\-\-Back (
\begin{DoxyParamCaption}
{}
\end{DoxyParamCaption}
)}}\label{class_list_a8b06ea3ceef6bb1b261656e78e1ba6e7}


Definicja w linii 101 pliku lista.\-cpp.



Oto graf wywoływań tej funkcji\-:


\hypertarget{class_list_a60d28fbb02bd3fc770ba0627d9345dde}{\index{List@{List}!Pop\-\_\-\-Front@{Pop\-\_\-\-Front}}
\index{Pop\-\_\-\-Front@{Pop\-\_\-\-Front}!List@{List}}
\subsubsection[{Pop\-\_\-\-Front}]{\setlength{\rightskip}{0pt plus 5cm}double List\-::\-Pop\-\_\-\-Front (
\begin{DoxyParamCaption}
{}
\end{DoxyParamCaption}
)}}\label{class_list_a60d28fbb02bd3fc770ba0627d9345dde}


Definicja w linii 79 pliku lista.\-cpp.



Oto graf wywoływań tej funkcji\-:


\hypertarget{class_list_a31fbd443a2454901d82e4baa1732fe62}{\index{List@{List}!Push\-\_\-\-Front@{Push\-\_\-\-Front}}
\index{Push\-\_\-\-Front@{Push\-\_\-\-Front}!List@{List}}
\subsubsection[{Push\-\_\-\-Front}]{\setlength{\rightskip}{0pt plus 5cm}void List\-::\-Push\-\_\-\-Front (
\begin{DoxyParamCaption}
\item[{double}]{number}
\end{DoxyParamCaption}
)}}\label{class_list_a31fbd443a2454901d82e4baa1732fe62}


Definicja w linii 58 pliku lista.\-cpp.



Oto graf wywoływań tej funkcji\-:


\hypertarget{class_list_a25ab387de5733d3a908b730877b0f260}{\index{List@{List}!Show\-\_\-\-List@{Show\-\_\-\-List}}
\index{Show\-\_\-\-List@{Show\-\_\-\-List}!List@{List}}
\subsubsection[{Show\-\_\-\-List}]{\setlength{\rightskip}{0pt plus 5cm}void List\-::\-Show\-\_\-\-List (
\begin{DoxyParamCaption}
{}
\end{DoxyParamCaption}
)}}\label{class_list_a25ab387de5733d3a908b730877b0f260}


Definicja w linii 126 pliku lista.\-cpp.



Oto graf wywoływań tej funkcji\-:


\hypertarget{class_list_a00e0054a58302c9eceb94d2ca884e6c5}{\index{List@{List}!Size@{Size}}
\index{Size@{Size}!List@{List}}
\subsubsection[{Size}]{\setlength{\rightskip}{0pt plus 5cm}int List\-::\-Size (
\begin{DoxyParamCaption}
{}
\end{DoxyParamCaption}
)}}\label{class_list_a00e0054a58302c9eceb94d2ca884e6c5}


Definicja w linii 47 pliku lista.\-cpp.



Oto graf wywoływań tej funkcji\-:




\subsection{Dokumentacja atrybutów składowych}
\hypertarget{class_list_ab64659e8fecbb40c66a0f63e6b6e14cb}{\index{List@{List}!back@{back}}
\index{back@{back}!List@{List}}
\subsubsection[{back}]{\setlength{\rightskip}{0pt plus 5cm}{\bf List\-Ele} $\ast$ List\-::back\hspace{0.3cm}{\ttfamily [private]}}}\label{class_list_ab64659e8fecbb40c66a0f63e6b6e14cb}


Definicja w linii 24 pliku lista.\-hpp.

\hypertarget{class_list_aa9b0031d0c9e9e4805c021fbbcc41b14}{\index{List@{List}!counter@{counter}}
\index{counter@{counter}!List@{List}}
\subsubsection[{counter}]{\setlength{\rightskip}{0pt plus 5cm}unsigned List\-::counter\hspace{0.3cm}{\ttfamily [private]}}}\label{class_list_aa9b0031d0c9e9e4805c021fbbcc41b14}


Definicja w linii 25 pliku lista.\-hpp.

\hypertarget{class_list_a1bca2e220bb62fc97b3d2952ee6aadec}{\index{List@{List}!front@{front}}
\index{front@{front}!List@{List}}
\subsubsection[{front}]{\setlength{\rightskip}{0pt plus 5cm}{\bf List\-Ele}$\ast$ List\-::front\hspace{0.3cm}{\ttfamily [private]}}}\label{class_list_a1bca2e220bb62fc97b3d2952ee6aadec}


Definicja w linii 24 pliku lista.\-hpp.



Dokumentacja dla tej klasy została wygenerowana z plików\-:\begin{DoxyCompactItemize}
\item 
/home/karolina/\-Pulpit/\-L\-A\-B3nowe/prj/\hyperlink{lista_8hpp}{lista.\-hpp}\item 
/home/karolina/\-Pulpit/\-L\-A\-B3nowe/prj/\hyperlink{lista_8cpp}{lista.\-cpp}\end{DoxyCompactItemize}

\hypertarget{struct_list_ele}{\section{Dokumentacja struktury List\-Ele}
\label{struct_list_ele}\index{List\-Ele@{List\-Ele}}
}


{\ttfamily \#include $<$lista.\-hpp$>$}



Diagram współpracy dla List\-Ele\-:
\subsection*{Atrybuty publiczne}
\begin{DoxyCompactItemize}
\item 
\hyperlink{struct_list_ele}{List\-Ele} $\ast$ \hyperlink{struct_list_ele_a94ce43c22a26ca3da57cdd34cd9005ad}{next}
\item 
double \hyperlink{struct_list_ele_a52bbe3635f6c0d0b095d12b97c0ee55c}{dane}
\end{DoxyCompactItemize}


\subsection{Opis szczegółowy}


Definicja w linii 10 pliku lista.\-hpp.



\subsection{Dokumentacja atrybutów składowych}
\hypertarget{struct_list_ele_a52bbe3635f6c0d0b095d12b97c0ee55c}{\index{List\-Ele@{List\-Ele}!dane@{dane}}
\index{dane@{dane}!ListEle@{List\-Ele}}
\subsubsection[{dane}]{\setlength{\rightskip}{0pt plus 5cm}double List\-Ele\-::dane}}\label{struct_list_ele_a52bbe3635f6c0d0b095d12b97c0ee55c}


Definicja w linii 13 pliku lista.\-hpp.

\hypertarget{struct_list_ele_a94ce43c22a26ca3da57cdd34cd9005ad}{\index{List\-Ele@{List\-Ele}!next@{next}}
\index{next@{next}!ListEle@{List\-Ele}}
\subsubsection[{next}]{\setlength{\rightskip}{0pt plus 5cm}{\bf List\-Ele}$\ast$ List\-Ele\-::next}}\label{struct_list_ele_a94ce43c22a26ca3da57cdd34cd9005ad}


Definicja w linii 12 pliku lista.\-hpp.



Dokumentacja dla tej struktury została wygenerowana z pliku\-:\begin{DoxyCompactItemize}
\item 
/home/karolina/\-Pulpit/\-L\-A\-B3nowe/prj/\hyperlink{lista_8hpp}{lista.\-hpp}\end{DoxyCompactItemize}

\hypertarget{class_queue}{\section{Dokumentacja klasy Queue}
\label{class_queue}\index{Queue@{Queue}}
}


{\ttfamily \#include $<$kolejka.\-hpp$>$}



Diagram współpracy dla Queue\-:
\subsection*{Metody publiczne}
\begin{DoxyCompactItemize}
\item 
\hyperlink{class_queue_af1fd9bf4e7e72c4393f70d6cc6510e72}{Queue} (int typ)
\item 
void \hyperlink{class_queue_ad79218b6296d87515f0a487764c44111}{Enqueue} (double ele)
\item 
double \hyperlink{class_queue_af6a908c687baa28ac3237dcf22c1ba13}{Dequeue} ()
\item 
void \hyperlink{class_queue_ae671ac7c20b47b9a57eb008e1b946bfe}{Isempty} ()
\item 
int \hyperlink{class_queue_a2b28fe3446577261546f74b7bbe3ccc6}{Size} ()
\item 
void \hyperlink{class_queue_a2f80b1ea8c0af424d0153af7563e1c34}{Show} ()
\item 
\hyperlink{class_queue}{Queue} \& \hyperlink{class_queue_aef9c898949be023f99190e5f45996587}{operator==} (\hyperlink{class_queue}{Queue} \&que)
\item 
\hyperlink{class_queue}{Queue} \& \hyperlink{class_queue_a1b806694ad653ed3dd0fc48a216f97bc}{operator$\ast$} (double M)
\end{DoxyCompactItemize}
\subsection*{Atrybuty prywatne}
\begin{DoxyCompactItemize}
\item 
\hyperlink{class_tablica}{Tablica} \hyperlink{class_queue_a54458992e4ee244ad98f283f1c553786}{tab}
\item 
\hyperlink{class_list}{List} \hyperlink{class_queue_a4260c29a224d41878c8f9665fbf793eb}{lista}
\item 
\hyperlink{stos_8hpp_ad7b974b79929c04ea204e3304ff8c776}{T\-Y\-P} \hyperlink{class_queue_a7636f4d2ef52ee88df5db9150f4b2b1d}{T}
\end{DoxyCompactItemize}


\subsection{Opis szczegółowy}


Definicja w linii 13 pliku kolejka.\-hpp.



\subsection{Dokumentacja konstruktora i destruktora}
\hypertarget{class_queue_af1fd9bf4e7e72c4393f70d6cc6510e72}{\index{Queue@{Queue}!Queue@{Queue}}
\index{Queue@{Queue}!Queue@{Queue}}
\subsubsection[{Queue}]{\setlength{\rightskip}{0pt plus 5cm}Queue\-::\-Queue (
\begin{DoxyParamCaption}
\item[{int}]{typ}
\end{DoxyParamCaption}
)\hspace{0.3cm}{\ttfamily [inline]}}}\label{class_queue_af1fd9bf4e7e72c4393f70d6cc6510e72}


Definicja w linii 22 pliku kolejka.\-hpp.



\subsection{Dokumentacja funkcji składowych}
\hypertarget{class_queue_af6a908c687baa28ac3237dcf22c1ba13}{\index{Queue@{Queue}!Dequeue@{Dequeue}}
\index{Dequeue@{Dequeue}!Queue@{Queue}}
\subsubsection[{Dequeue}]{\setlength{\rightskip}{0pt plus 5cm}double Queue\-::\-Dequeue (
\begin{DoxyParamCaption}
{}
\end{DoxyParamCaption}
)\hspace{0.3cm}{\ttfamily [inline]}}}\label{class_queue_af6a908c687baa28ac3237dcf22c1ba13}


Definicja w linii 25 pliku kolejka.\-hpp.



Oto graf wywołań dla tej funkcji\-:




Oto graf wywoływań tej funkcji\-:


\hypertarget{class_queue_ad79218b6296d87515f0a487764c44111}{\index{Queue@{Queue}!Enqueue@{Enqueue}}
\index{Enqueue@{Enqueue}!Queue@{Queue}}
\subsubsection[{Enqueue}]{\setlength{\rightskip}{0pt plus 5cm}void Queue\-::\-Enqueue (
\begin{DoxyParamCaption}
\item[{double}]{ele}
\end{DoxyParamCaption}
)\hspace{0.3cm}{\ttfamily [inline]}}}\label{class_queue_ad79218b6296d87515f0a487764c44111}


Definicja w linii 24 pliku kolejka.\-hpp.



Oto graf wywołań dla tej funkcji\-:




Oto graf wywoływań tej funkcji\-:


\hypertarget{class_queue_ae671ac7c20b47b9a57eb008e1b946bfe}{\index{Queue@{Queue}!Isempty@{Isempty}}
\index{Isempty@{Isempty}!Queue@{Queue}}
\subsubsection[{Isempty}]{\setlength{\rightskip}{0pt plus 5cm}void Queue\-::\-Isempty (
\begin{DoxyParamCaption}
{}
\end{DoxyParamCaption}
)\hspace{0.3cm}{\ttfamily [inline]}}}\label{class_queue_ae671ac7c20b47b9a57eb008e1b946bfe}


Definicja w linii 26 pliku kolejka.\-hpp.



Oto graf wywołań dla tej funkcji\-:


\hypertarget{class_queue_a1b806694ad653ed3dd0fc48a216f97bc}{\index{Queue@{Queue}!operator$\ast$@{operator$\ast$}}
\index{operator$\ast$@{operator$\ast$}!Queue@{Queue}}
\subsubsection[{operator$\ast$}]{\setlength{\rightskip}{0pt plus 5cm}{\bf Queue}\& Queue\-::operator$\ast$ (
\begin{DoxyParamCaption}
\item[{double}]{M}
\end{DoxyParamCaption}
)\hspace{0.3cm}{\ttfamily [inline]}}}\label{class_queue_a1b806694ad653ed3dd0fc48a216f97bc}


Definicja w linii 31 pliku kolejka.\-hpp.

\hypertarget{class_queue_aef9c898949be023f99190e5f45996587}{\index{Queue@{Queue}!operator==@{operator==}}
\index{operator==@{operator==}!Queue@{Queue}}
\subsubsection[{operator==}]{\setlength{\rightskip}{0pt plus 5cm}{\bf Queue}\& Queue\-::operator== (
\begin{DoxyParamCaption}
\item[{{\bf Queue} \&}]{que}
\end{DoxyParamCaption}
)\hspace{0.3cm}{\ttfamily [inline]}}}\label{class_queue_aef9c898949be023f99190e5f45996587}


Definicja w linii 30 pliku kolejka.\-hpp.

\hypertarget{class_queue_a2f80b1ea8c0af424d0153af7563e1c34}{\index{Queue@{Queue}!Show@{Show}}
\index{Show@{Show}!Queue@{Queue}}
\subsubsection[{Show}]{\setlength{\rightskip}{0pt plus 5cm}void Queue\-::\-Show (
\begin{DoxyParamCaption}
{}
\end{DoxyParamCaption}
)\hspace{0.3cm}{\ttfamily [inline]}}}\label{class_queue_a2f80b1ea8c0af424d0153af7563e1c34}


Definicja w linii 29 pliku kolejka.\-hpp.



Oto graf wywołań dla tej funkcji\-:


\hypertarget{class_queue_a2b28fe3446577261546f74b7bbe3ccc6}{\index{Queue@{Queue}!Size@{Size}}
\index{Size@{Size}!Queue@{Queue}}
\subsubsection[{Size}]{\setlength{\rightskip}{0pt plus 5cm}int Queue\-::\-Size (
\begin{DoxyParamCaption}
{}
\end{DoxyParamCaption}
)\hspace{0.3cm}{\ttfamily [inline]}}}\label{class_queue_a2b28fe3446577261546f74b7bbe3ccc6}


Definicja w linii 27 pliku kolejka.\-hpp.



Oto graf wywołań dla tej funkcji\-:




Oto graf wywoływań tej funkcji\-:




\subsection{Dokumentacja atrybutów składowych}
\hypertarget{class_queue_a4260c29a224d41878c8f9665fbf793eb}{\index{Queue@{Queue}!lista@{lista}}
\index{lista@{lista}!Queue@{Queue}}
\subsubsection[{lista}]{\setlength{\rightskip}{0pt plus 5cm}{\bf List} Queue\-::lista\hspace{0.3cm}{\ttfamily [private]}}}\label{class_queue_a4260c29a224d41878c8f9665fbf793eb}


Definicja w linii 18 pliku kolejka.\-hpp.

\hypertarget{class_queue_a7636f4d2ef52ee88df5db9150f4b2b1d}{\index{Queue@{Queue}!T@{T}}
\index{T@{T}!Queue@{Queue}}
\subsubsection[{T}]{\setlength{\rightskip}{0pt plus 5cm}{\bf T\-Y\-P} Queue\-::\-T\hspace{0.3cm}{\ttfamily [private]}}}\label{class_queue_a7636f4d2ef52ee88df5db9150f4b2b1d}


Definicja w linii 19 pliku kolejka.\-hpp.

\hypertarget{class_queue_a54458992e4ee244ad98f283f1c553786}{\index{Queue@{Queue}!tab@{tab}}
\index{tab@{tab}!Queue@{Queue}}
\subsubsection[{tab}]{\setlength{\rightskip}{0pt plus 5cm}{\bf Tablica} Queue\-::tab\hspace{0.3cm}{\ttfamily [private]}}}\label{class_queue_a54458992e4ee244ad98f283f1c553786}


Definicja w linii 17 pliku kolejka.\-hpp.



Dokumentacja dla tej klasy została wygenerowana z pliku\-:\begin{DoxyCompactItemize}
\item 
/home/karolina/\-Pulpit/\-L\-A\-B3nowe/prj/\hyperlink{kolejka_8hpp}{kolejka.\-hpp}\end{DoxyCompactItemize}

\hypertarget{class_stos}{\section{Dokumentacja klasy Stos}
\label{class_stos}\index{Stos@{Stos}}
}


{\ttfamily \#include $<$stos.\-hpp$>$}



Diagram współpracy dla Stos\-:
\subsection*{Metody publiczne}
\begin{DoxyCompactItemize}
\item 
\hyperlink{class_stos_a9f1a0e5eb781e7296493d9c61f6ce2e2}{Stos} (int typ)
\item 
void \hyperlink{class_stos_af8c0c485dceb28e986cdc93e933ec365}{Push} (double ele)
\item 
double \hyperlink{class_stos_a620d6e49f74becaed97de2a911ef051c}{Pop} ()
\item 
void \hyperlink{class_stos_a8e559abb36abb64c458bdc281e273cba}{Isempty} ()
\item 
int \hyperlink{class_stos_a9861291fa460528db0dea0262bf88f69}{Size} ()
\item 
void \hyperlink{class_stos_abfb03d80298013a2792bde2fd89c3863}{Show} ()
\item 
\hyperlink{class_stos}{Stos} \& \hyperlink{class_stos_a46c03bf71cc3b8abc5537a5a408c43e8}{operator==} (\hyperlink{class_stos}{Stos} \&sto)
\item 
\hyperlink{class_stos}{Stos} \& \hyperlink{class_stos_a1a82883d837c3c63c3621e284c707304}{operator$\ast$} (double M)
\end{DoxyCompactItemize}
\subsection*{Atrybuty prywatne}
\begin{DoxyCompactItemize}
\item 
\hyperlink{class_tablica}{Tablica} \hyperlink{class_stos_a0cdca9595d13e2563eb3e0230ecf7619}{tab}
\item 
\hyperlink{class_list}{List} \hyperlink{class_stos_aad8157be87d5c0266b951b7e2bae7e5d}{lista}
\item 
\hyperlink{stos_8hpp_ad7b974b79929c04ea204e3304ff8c776}{T\-Y\-P} \hyperlink{class_stos_a62fd3591c5613480c3c770d90f7b91b1}{T}
\end{DoxyCompactItemize}


\subsection{Opis szczegółowy}


Definicja w linii 15 pliku stos.\-hpp.



\subsection{Dokumentacja konstruktora i destruktora}
\hypertarget{class_stos_a9f1a0e5eb781e7296493d9c61f6ce2e2}{\index{Stos@{Stos}!Stos@{Stos}}
\index{Stos@{Stos}!Stos@{Stos}}
\subsubsection[{Stos}]{\setlength{\rightskip}{0pt plus 5cm}Stos\-::\-Stos (
\begin{DoxyParamCaption}
\item[{int}]{typ}
\end{DoxyParamCaption}
)\hspace{0.3cm}{\ttfamily [inline]}}}\label{class_stos_a9f1a0e5eb781e7296493d9c61f6ce2e2}


Definicja w linii 24 pliku stos.\-hpp.



\subsection{Dokumentacja funkcji składowych}
\hypertarget{class_stos_a8e559abb36abb64c458bdc281e273cba}{\index{Stos@{Stos}!Isempty@{Isempty}}
\index{Isempty@{Isempty}!Stos@{Stos}}
\subsubsection[{Isempty}]{\setlength{\rightskip}{0pt plus 5cm}void Stos\-::\-Isempty (
\begin{DoxyParamCaption}
{}
\end{DoxyParamCaption}
)\hspace{0.3cm}{\ttfamily [inline]}}}\label{class_stos_a8e559abb36abb64c458bdc281e273cba}


Definicja w linii 28 pliku stos.\-hpp.



Oto graf wywołań dla tej funkcji\-:


\hypertarget{class_stos_a1a82883d837c3c63c3621e284c707304}{\index{Stos@{Stos}!operator$\ast$@{operator$\ast$}}
\index{operator$\ast$@{operator$\ast$}!Stos@{Stos}}
\subsubsection[{operator$\ast$}]{\setlength{\rightskip}{0pt plus 5cm}{\bf Stos}\& Stos\-::operator$\ast$ (
\begin{DoxyParamCaption}
\item[{double}]{M}
\end{DoxyParamCaption}
)\hspace{0.3cm}{\ttfamily [inline]}}}\label{class_stos_a1a82883d837c3c63c3621e284c707304}


Definicja w linii 33 pliku stos.\-hpp.

\hypertarget{class_stos_a46c03bf71cc3b8abc5537a5a408c43e8}{\index{Stos@{Stos}!operator==@{operator==}}
\index{operator==@{operator==}!Stos@{Stos}}
\subsubsection[{operator==}]{\setlength{\rightskip}{0pt plus 5cm}{\bf Stos}\& Stos\-::operator== (
\begin{DoxyParamCaption}
\item[{{\bf Stos} \&}]{sto}
\end{DoxyParamCaption}
)\hspace{0.3cm}{\ttfamily [inline]}}}\label{class_stos_a46c03bf71cc3b8abc5537a5a408c43e8}


Definicja w linii 32 pliku stos.\-hpp.

\hypertarget{class_stos_a620d6e49f74becaed97de2a911ef051c}{\index{Stos@{Stos}!Pop@{Pop}}
\index{Pop@{Pop}!Stos@{Stos}}
\subsubsection[{Pop}]{\setlength{\rightskip}{0pt plus 5cm}double Stos\-::\-Pop (
\begin{DoxyParamCaption}
{}
\end{DoxyParamCaption}
)\hspace{0.3cm}{\ttfamily [inline]}}}\label{class_stos_a620d6e49f74becaed97de2a911ef051c}


Definicja w linii 27 pliku stos.\-hpp.



Oto graf wywołań dla tej funkcji\-:




Oto graf wywoływań tej funkcji\-:


\hypertarget{class_stos_af8c0c485dceb28e986cdc93e933ec365}{\index{Stos@{Stos}!Push@{Push}}
\index{Push@{Push}!Stos@{Stos}}
\subsubsection[{Push}]{\setlength{\rightskip}{0pt plus 5cm}void Stos\-::\-Push (
\begin{DoxyParamCaption}
\item[{double}]{ele}
\end{DoxyParamCaption}
)\hspace{0.3cm}{\ttfamily [inline]}}}\label{class_stos_af8c0c485dceb28e986cdc93e933ec365}


Definicja w linii 26 pliku stos.\-hpp.



Oto graf wywołań dla tej funkcji\-:




Oto graf wywoływań tej funkcji\-:


\hypertarget{class_stos_abfb03d80298013a2792bde2fd89c3863}{\index{Stos@{Stos}!Show@{Show}}
\index{Show@{Show}!Stos@{Stos}}
\subsubsection[{Show}]{\setlength{\rightskip}{0pt plus 5cm}void Stos\-::\-Show (
\begin{DoxyParamCaption}
{}
\end{DoxyParamCaption}
)\hspace{0.3cm}{\ttfamily [inline]}}}\label{class_stos_abfb03d80298013a2792bde2fd89c3863}


Definicja w linii 31 pliku stos.\-hpp.



Oto graf wywołań dla tej funkcji\-:


\hypertarget{class_stos_a9861291fa460528db0dea0262bf88f69}{\index{Stos@{Stos}!Size@{Size}}
\index{Size@{Size}!Stos@{Stos}}
\subsubsection[{Size}]{\setlength{\rightskip}{0pt plus 5cm}int Stos\-::\-Size (
\begin{DoxyParamCaption}
{}
\end{DoxyParamCaption}
)\hspace{0.3cm}{\ttfamily [inline]}}}\label{class_stos_a9861291fa460528db0dea0262bf88f69}


Definicja w linii 29 pliku stos.\-hpp.



Oto graf wywołań dla tej funkcji\-:




Oto graf wywoływań tej funkcji\-:




\subsection{Dokumentacja atrybutów składowych}
\hypertarget{class_stos_aad8157be87d5c0266b951b7e2bae7e5d}{\index{Stos@{Stos}!lista@{lista}}
\index{lista@{lista}!Stos@{Stos}}
\subsubsection[{lista}]{\setlength{\rightskip}{0pt plus 5cm}{\bf List} Stos\-::lista\hspace{0.3cm}{\ttfamily [private]}}}\label{class_stos_aad8157be87d5c0266b951b7e2bae7e5d}


Definicja w linii 20 pliku stos.\-hpp.

\hypertarget{class_stos_a62fd3591c5613480c3c770d90f7b91b1}{\index{Stos@{Stos}!T@{T}}
\index{T@{T}!Stos@{Stos}}
\subsubsection[{T}]{\setlength{\rightskip}{0pt plus 5cm}{\bf T\-Y\-P} Stos\-::\-T\hspace{0.3cm}{\ttfamily [private]}}}\label{class_stos_a62fd3591c5613480c3c770d90f7b91b1}


Definicja w linii 21 pliku stos.\-hpp.

\hypertarget{class_stos_a0cdca9595d13e2563eb3e0230ecf7619}{\index{Stos@{Stos}!tab@{tab}}
\index{tab@{tab}!Stos@{Stos}}
\subsubsection[{tab}]{\setlength{\rightskip}{0pt plus 5cm}{\bf Tablica} Stos\-::tab\hspace{0.3cm}{\ttfamily [private]}}}\label{class_stos_a0cdca9595d13e2563eb3e0230ecf7619}


Definicja w linii 19 pliku stos.\-hpp.



Dokumentacja dla tej klasy została wygenerowana z pliku\-:\begin{DoxyCompactItemize}
\item 
/home/karolina/\-Pulpit/\-L\-A\-B3nowe/prj/\hyperlink{stos_8hpp}{stos.\-hpp}\end{DoxyCompactItemize}

\hypertarget{class_tablica}{\section{Dokumentacja klasy Tablica}
\label{class_tablica}\index{Tablica@{Tablica}}
}


{\ttfamily \#include $<$tablica.\-hpp$>$}

\subsection*{Metody publiczne}
\begin{DoxyCompactItemize}
\item 
\hyperlink{class_tablica_a5f484e7b0478e1ff9b62e894f9d7b28d}{Tablica} ()
\item 
void \hyperlink{class_tablica_ae1af903a66629cd0d522eb9f2fd13116}{Push} (double ele)
\item 
double \hyperlink{class_tablica_a6153881ffda3f5361c2d664622a4eff4}{Pop} ()
\item 
double \hyperlink{class_tablica_a899c8e69cb97bd027c1c05140cd304ec}{Pop\-\_\-\-Back} ()
\item 
int \hyperlink{class_tablica_a8598f952095406441bfd2d20e76f175c}{Size} ()
\item 
void \hyperlink{class_tablica_a08b59415756d2dc7da781124809d8eb4}{Isempty} ()
\item 
void \hyperlink{class_tablica_a06c551a7e0220dde2f29cce06fb96209}{Show\-\_\-\-Tab} ()
\item 
\hyperlink{class_tablica}{Tablica} \hyperlink{class_tablica_a15c072e7160bfbdbc5d103cf0ebd6e76}{operator+} (\hyperlink{class_tablica}{Tablica} \&T2)
\item 
\hyperlink{class_tablica}{Tablica} \hyperlink{class_tablica_acabfd453d919950051b8a9cf4aac642e}{operator$\ast$} (double M)
\item 
\hyperlink{class_tablica}{Tablica} \hyperlink{class_tablica_a53bd7c9853f01a78ba2aff61ece4ccbf}{operator=} (\hyperlink{class_tablica}{Tablica} \&T2)
\item 
\hyperlink{class_tablica}{Tablica} \hyperlink{class_tablica_ae5d9fdf31df882eae683abc89fec01ad}{operator==} (\hyperlink{class_tablica}{Tablica} \&T2)
\end{DoxyCompactItemize}
\subsection*{Atrybuty prywatne}
\begin{DoxyCompactItemize}
\item 
vector$<$ double $>$ \hyperlink{class_tablica_aaa8c514d3b76071eefab059a49de706a}{tab}
\end{DoxyCompactItemize}


\subsection{Opis szczegółowy}


Definicja w linii 14 pliku tablica.\-hpp.



\subsection{Dokumentacja konstruktora i destruktora}
\hypertarget{class_tablica_a5f484e7b0478e1ff9b62e894f9d7b28d}{\index{Tablica@{Tablica}!Tablica@{Tablica}}
\index{Tablica@{Tablica}!Tablica@{Tablica}}
\subsubsection[{Tablica}]{\setlength{\rightskip}{0pt plus 5cm}Tablica\-::\-Tablica (
\begin{DoxyParamCaption}
{}
\end{DoxyParamCaption}
)\hspace{0.3cm}{\ttfamily [inline]}}}\label{class_tablica_a5f484e7b0478e1ff9b62e894f9d7b28d}


Definicja w linii 20 pliku tablica.\-hpp.



\subsection{Dokumentacja funkcji składowych}
\hypertarget{class_tablica_a08b59415756d2dc7da781124809d8eb4}{\index{Tablica@{Tablica}!Isempty@{Isempty}}
\index{Isempty@{Isempty}!Tablica@{Tablica}}
\subsubsection[{Isempty}]{\setlength{\rightskip}{0pt plus 5cm}void Tablica\-::\-Isempty (
\begin{DoxyParamCaption}
{}
\end{DoxyParamCaption}
)}}\label{class_tablica_a08b59415756d2dc7da781124809d8eb4}


Definicja w linii 108 pliku tablica.\-cpp.



Oto graf wywoływań tej funkcji\-:


\hypertarget{class_tablica_acabfd453d919950051b8a9cf4aac642e}{\index{Tablica@{Tablica}!operator$\ast$@{operator$\ast$}}
\index{operator$\ast$@{operator$\ast$}!Tablica@{Tablica}}
\subsubsection[{operator$\ast$}]{\setlength{\rightskip}{0pt plus 5cm}{\bf Tablica} Tablica\-::operator$\ast$ (
\begin{DoxyParamCaption}
\item[{double}]{M}
\end{DoxyParamCaption}
)}}\label{class_tablica_acabfd453d919950051b8a9cf4aac642e}


Definicja w linii 273 pliku tablica.\-cpp.

\hypertarget{class_tablica_a15c072e7160bfbdbc5d103cf0ebd6e76}{\index{Tablica@{Tablica}!operator+@{operator+}}
\index{operator+@{operator+}!Tablica@{Tablica}}
\subsubsection[{operator+}]{\setlength{\rightskip}{0pt plus 5cm}{\bf Tablica} Tablica\-::operator+ (
\begin{DoxyParamCaption}
\item[{{\bf Tablica} \&}]{T2}
\end{DoxyParamCaption}
)}}\label{class_tablica_a15c072e7160bfbdbc5d103cf0ebd6e76}


Definicja w linii 203 pliku tablica.\-cpp.

\hypertarget{class_tablica_a53bd7c9853f01a78ba2aff61ece4ccbf}{\index{Tablica@{Tablica}!operator=@{operator=}}
\index{operator=@{operator=}!Tablica@{Tablica}}
\subsubsection[{operator=}]{\setlength{\rightskip}{0pt plus 5cm}{\bf Tablica} Tablica\-::operator= (
\begin{DoxyParamCaption}
\item[{{\bf Tablica} \&}]{T2}
\end{DoxyParamCaption}
)}}\label{class_tablica_a53bd7c9853f01a78ba2aff61ece4ccbf}


Definicja w linii 218 pliku tablica.\-cpp.

\hypertarget{class_tablica_ae5d9fdf31df882eae683abc89fec01ad}{\index{Tablica@{Tablica}!operator==@{operator==}}
\index{operator==@{operator==}!Tablica@{Tablica}}
\subsubsection[{operator==}]{\setlength{\rightskip}{0pt plus 5cm}{\bf Tablica} Tablica\-::operator== (
\begin{DoxyParamCaption}
\item[{{\bf Tablica} \&}]{T2}
\end{DoxyParamCaption}
)}}\label{class_tablica_ae5d9fdf31df882eae683abc89fec01ad}


Definicja w linii 229 pliku tablica.\-cpp.

\hypertarget{class_tablica_a6153881ffda3f5361c2d664622a4eff4}{\index{Tablica@{Tablica}!Pop@{Pop}}
\index{Pop@{Pop}!Tablica@{Tablica}}
\subsubsection[{Pop}]{\setlength{\rightskip}{0pt plus 5cm}double Tablica\-::\-Pop (
\begin{DoxyParamCaption}
{}
\end{DoxyParamCaption}
)}}\label{class_tablica_a6153881ffda3f5361c2d664622a4eff4}


Definicja w linii 31 pliku tablica.\-cpp.



Oto graf wywoływań tej funkcji\-:


\hypertarget{class_tablica_a899c8e69cb97bd027c1c05140cd304ec}{\index{Tablica@{Tablica}!Pop\-\_\-\-Back@{Pop\-\_\-\-Back}}
\index{Pop\-\_\-\-Back@{Pop\-\_\-\-Back}!Tablica@{Tablica}}
\subsubsection[{Pop\-\_\-\-Back}]{\setlength{\rightskip}{0pt plus 5cm}double Tablica\-::\-Pop\-\_\-\-Back (
\begin{DoxyParamCaption}
{}
\end{DoxyParamCaption}
)}}\label{class_tablica_a899c8e69cb97bd027c1c05140cd304ec}


Definicja w linii 64 pliku tablica.\-cpp.



Oto graf wywoływań tej funkcji\-:


\hypertarget{class_tablica_ae1af903a66629cd0d522eb9f2fd13116}{\index{Tablica@{Tablica}!Push@{Push}}
\index{Push@{Push}!Tablica@{Tablica}}
\subsubsection[{Push}]{\setlength{\rightskip}{0pt plus 5cm}void Tablica\-::\-Push (
\begin{DoxyParamCaption}
\item[{double}]{ele}
\end{DoxyParamCaption}
)}}\label{class_tablica_ae1af903a66629cd0d522eb9f2fd13116}


Definicja w linii 16 pliku tablica.\-cpp.



Oto graf wywoływań tej funkcji\-:


\hypertarget{class_tablica_a06c551a7e0220dde2f29cce06fb96209}{\index{Tablica@{Tablica}!Show\-\_\-\-Tab@{Show\-\_\-\-Tab}}
\index{Show\-\_\-\-Tab@{Show\-\_\-\-Tab}!Tablica@{Tablica}}
\subsubsection[{Show\-\_\-\-Tab}]{\setlength{\rightskip}{0pt plus 5cm}void Tablica\-::\-Show\-\_\-\-Tab (
\begin{DoxyParamCaption}
{}
\end{DoxyParamCaption}
)}}\label{class_tablica_a06c551a7e0220dde2f29cce06fb96209}


Definicja w linii 177 pliku tablica.\-cpp.



Oto graf wywoływań tej funkcji\-:


\hypertarget{class_tablica_a8598f952095406441bfd2d20e76f175c}{\index{Tablica@{Tablica}!Size@{Size}}
\index{Size@{Size}!Tablica@{Tablica}}
\subsubsection[{Size}]{\setlength{\rightskip}{0pt plus 5cm}int Tablica\-::\-Size (
\begin{DoxyParamCaption}
{}
\end{DoxyParamCaption}
)}}\label{class_tablica_a8598f952095406441bfd2d20e76f175c}


Definicja w linii 96 pliku tablica.\-cpp.



Oto graf wywoływań tej funkcji\-:




\subsection{Dokumentacja atrybutów składowych}
\hypertarget{class_tablica_aaa8c514d3b76071eefab059a49de706a}{\index{Tablica@{Tablica}!tab@{tab}}
\index{tab@{tab}!Tablica@{Tablica}}
\subsubsection[{tab}]{\setlength{\rightskip}{0pt plus 5cm}vector$<$ double $>$ Tablica\-::tab\hspace{0.3cm}{\ttfamily [private]}}}\label{class_tablica_aaa8c514d3b76071eefab059a49de706a}


Definicja w linii 17 pliku tablica.\-hpp.



Dokumentacja dla tej klasy została wygenerowana z plików\-:\begin{DoxyCompactItemize}
\item 
/home/karolina/\-Pulpit/\-L\-A\-B3nowe/prj/\hyperlink{tablica_8hpp}{tablica.\-hpp}\item 
/home/karolina/\-Pulpit/\-L\-A\-B3nowe/prj/\hyperlink{tablica_8cpp}{tablica.\-cpp}\end{DoxyCompactItemize}

\hypertarget{class_zegar}{\section{Dokumentacja klasy Zegar}
\label{class_zegar}\index{Zegar@{Zegar}}
}


{\ttfamily \#include $<$zegar.\-hpp$>$}

\subsection*{Metody publiczne}
\begin{DoxyCompactItemize}
\item 
void \hyperlink{class_zegar_af747dc3a9d58207618ec877990900b80}{Start} ()
\item 
void \hyperlink{class_zegar_a8a88ddd1aa0768bfbe37217e32a01da0}{Koniec} ()
\item 
int \hyperlink{class_zegar_a5398289b65c5c6f2fa87fde6d48ab4dd}{Wynik} ()
\end{DoxyCompactItemize}
\subsection*{Atrybuty prywatne}
\begin{DoxyCompactItemize}
\item 
clock\-\_\-t \hyperlink{class_zegar_a7486e73fa453ccef2a597b2b7c9d0d20}{start}
\item 
clock\-\_\-t \hyperlink{class_zegar_ae74d610e2ffed2f50f16349e21cc2ef8}{koniec}
\item 
double \hyperlink{class_zegar_ac70530370d36285da9a2ebee2530f6cc}{czas}
\end{DoxyCompactItemize}


\subsection{Opis szczegółowy}


Definicja w linii 16 pliku zegar.\-hpp.



\subsection{Dokumentacja funkcji składowych}
\hypertarget{class_zegar_a8a88ddd1aa0768bfbe37217e32a01da0}{\index{Zegar@{Zegar}!Koniec@{Koniec}}
\index{Koniec@{Koniec}!Zegar@{Zegar}}
\subsubsection[{Koniec}]{\setlength{\rightskip}{0pt plus 5cm}void Zegar\-::\-Koniec (
\begin{DoxyParamCaption}
{}
\end{DoxyParamCaption}
)}}\label{class_zegar_a8a88ddd1aa0768bfbe37217e32a01da0}


Definicja w linii 20 pliku zegar.\-cpp.



Oto graf wywoływań tej funkcji\-:


\hypertarget{class_zegar_af747dc3a9d58207618ec877990900b80}{\index{Zegar@{Zegar}!Start@{Start}}
\index{Start@{Start}!Zegar@{Zegar}}
\subsubsection[{Start}]{\setlength{\rightskip}{0pt plus 5cm}void Zegar\-::\-Start (
\begin{DoxyParamCaption}
{}
\end{DoxyParamCaption}
)}}\label{class_zegar_af747dc3a9d58207618ec877990900b80}


Definicja w linii 9 pliku zegar.\-cpp.



Oto graf wywoływań tej funkcji\-:


\hypertarget{class_zegar_a5398289b65c5c6f2fa87fde6d48ab4dd}{\index{Zegar@{Zegar}!Wynik@{Wynik}}
\index{Wynik@{Wynik}!Zegar@{Zegar}}
\subsubsection[{Wynik}]{\setlength{\rightskip}{0pt plus 5cm}int Zegar\-::\-Wynik (
\begin{DoxyParamCaption}
{}
\end{DoxyParamCaption}
)}}\label{class_zegar_a5398289b65c5c6f2fa87fde6d48ab4dd}


Definicja w linii 36 pliku zegar.\-cpp.



Oto graf wywoływań tej funkcji\-:




\subsection{Dokumentacja atrybutów składowych}
\hypertarget{class_zegar_ac70530370d36285da9a2ebee2530f6cc}{\index{Zegar@{Zegar}!czas@{czas}}
\index{czas@{czas}!Zegar@{Zegar}}
\subsubsection[{czas}]{\setlength{\rightskip}{0pt plus 5cm}double Zegar\-::czas\hspace{0.3cm}{\ttfamily [private]}}}\label{class_zegar_ac70530370d36285da9a2ebee2530f6cc}


Definicja w linii 20 pliku zegar.\-hpp.

\hypertarget{class_zegar_ae74d610e2ffed2f50f16349e21cc2ef8}{\index{Zegar@{Zegar}!koniec@{koniec}}
\index{koniec@{koniec}!Zegar@{Zegar}}
\subsubsection[{koniec}]{\setlength{\rightskip}{0pt plus 5cm}clock\-\_\-t Zegar\-::koniec\hspace{0.3cm}{\ttfamily [private]}}}\label{class_zegar_ae74d610e2ffed2f50f16349e21cc2ef8}


Definicja w linii 19 pliku zegar.\-hpp.

\hypertarget{class_zegar_a7486e73fa453ccef2a597b2b7c9d0d20}{\index{Zegar@{Zegar}!start@{start}}
\index{start@{start}!Zegar@{Zegar}}
\subsubsection[{start}]{\setlength{\rightskip}{0pt plus 5cm}clock\-\_\-t Zegar\-::start\hspace{0.3cm}{\ttfamily [private]}}}\label{class_zegar_a7486e73fa453ccef2a597b2b7c9d0d20}


Definicja w linii 19 pliku zegar.\-hpp.



Dokumentacja dla tej klasy została wygenerowana z plików\-:\begin{DoxyCompactItemize}
\item 
/home/karolina/\-Pulpit/\-L\-A\-B3nowe/prj/\hyperlink{zegar_8hpp}{zegar.\-hpp}\item 
/home/karolina/\-Pulpit/\-L\-A\-B3nowe/prj/\hyperlink{zegar_8cpp}{zegar.\-cpp}\end{DoxyCompactItemize}

\chapter{Dokumentacja plików}
\hypertarget{kolejka_8hpp}{\section{Dokumentacja pliku /home/karolina/\-Pulpit/\-L\-A\-B3nowe/prj/kolejka.hpp}
\label{kolejka_8hpp}\index{/home/karolina/\-Pulpit/\-L\-A\-B3nowe/prj/kolejka.\-hpp@{/home/karolina/\-Pulpit/\-L\-A\-B3nowe/prj/kolejka.\-hpp}}
}


Definicja klasy \hyperlink{class_queue}{Queue}.  


{\ttfamily \#include \char`\"{}tablica.\-hpp\char`\"{}}\\*
{\ttfamily \#include \char`\"{}lista.\-hpp\char`\"{}}\\*
Wykres zależności załączania dla kolejka.\-hpp\-:
Ten wykres pokazuje, które pliki bezpośrednio lub pośrednio załączają ten plik\-:
\subsection*{Komponenty}
\begin{DoxyCompactItemize}
\item 
class \hyperlink{class_queue}{Queue}
\end{DoxyCompactItemize}


\subsection{Opis szczegółowy}
Definicja klasy \hyperlink{class_queue}{Queue}. Klasa odpowiadaj�ca za metody wykonywane na tablicy takie jak\-: Enqueue, Dequeue, Isempty, Size, Show, operatory\-: == $\ast$ 

Definicja w pliku \hyperlink{kolejka_8hpp_source}{kolejka.\-hpp}.


\hypertarget{lista_8cpp}{\section{Dokumentacja pliku /home/karolina/\-Pulpit/\-L\-A\-B3nowe/prj/lista.cpp}
\label{lista_8cpp}\index{/home/karolina/\-Pulpit/\-L\-A\-B3nowe/prj/lista.\-cpp@{/home/karolina/\-Pulpit/\-L\-A\-B3nowe/prj/lista.\-cpp}}
}


Definicja konstruktora \hyperlink{class_list}{List}.  


{\ttfamily \#include \char`\"{}lista.\-hpp\char`\"{}}\\*
Wykres zależności załączania dla lista.\-cpp\-:


\subsection{Opis szczegółowy}
Definicja konstruktora \hyperlink{class_list}{List}. Przeciazenie operatora$\ast$.

Przeciazenie operatora ==.

Definicja konstruktora Show\-\_\-\-List.

Definicja konstruktora Pop\-\_\-\-Back.

Definicja konstruktora Pop\-\_\-\-Front.

Definicja konstruktora Push\-\_\-\-Front.

Definicja konstruktora Size.

Definicja konstruktora Isempty.

Definicja destruktora \hyperlink{class_list}{List}.

Prztwarza obiekt do znaczniku front i back, przypisuje N\-U\-L\-L i liczy ile el. ma lisya

Usuwa wszystkie elementy

Sprawdza czy lista jest pusta

Rozmiar listy

Dodawanie na poczatek

Jest to dla stosu. Dodaje na poczatek.

Jest to dla kolejki. Dodaje na koniec.

Pokazuje liste.

sprawdzanie czy nie sa puste, jezeli nie to\-: 1-\/ 1$>$2 wykonywany 1 if 2$>$1 drugi if else kiedy takie same

Przeciazenie operatora$\ast$ 

Definicja w pliku \hyperlink{lista_8cpp_source}{lista.\-cpp}.


\hypertarget{lista_8hpp}{\section{Dokumentacja pliku /home/karolina/\-Pulpit/\-L\-A\-B3nowe/prj/lista.hpp}
\label{lista_8hpp}\index{/home/karolina/\-Pulpit/\-L\-A\-B3nowe/prj/lista.\-hpp@{/home/karolina/\-Pulpit/\-L\-A\-B3nowe/prj/lista.\-hpp}}
}


Definicja Struktury \hyperlink{struct_list_ele}{List\-Ele}.  


{\ttfamily \#include $<$iostream$>$}\\*
Wykres zależności załączania dla lista.\-hpp\-:
Ten wykres pokazuje, które pliki bezpośrednio lub pośrednio załączają ten plik\-:
\subsection*{Komponenty}
\begin{DoxyCompactItemize}
\item 
struct \hyperlink{struct_list_ele}{List\-Ele}
\item 
class \hyperlink{class_list}{List}
\end{DoxyCompactItemize}


\subsection{Opis szczegółowy}
Definicja Struktury \hyperlink{struct_list_ele}{List\-Ele}. Definicja klasy Lista.

Struktura, w kt�rej zdefiniowany jest wskaznik na kolejny element listy oraz zmienna dane, ktora przechowuje dane w tym elemencie

Klasa zawira wska�niki na 1 i ostani element listy )(front, back) oraz metody takie jak\-: is empty, size, push, pop show. 

Definicja w pliku \hyperlink{lista_8hpp_source}{lista.\-hpp}.


\hypertarget{main_8cpp}{\section{Dokumentacja pliku /home/karolina/\-Pulpit/\-L\-A\-B3nowe/prj/main.cpp}
\label{main_8cpp}\index{/home/karolina/\-Pulpit/\-L\-A\-B3nowe/prj/main.\-cpp@{/home/karolina/\-Pulpit/\-L\-A\-B3nowe/prj/main.\-cpp}}
}
{\ttfamily \#include $<$iostream$>$}\\*
{\ttfamily \#include \char`\"{}zegar.\-hpp\char`\"{}}\\*
{\ttfamily \#include \char`\"{}stos.\-hpp\char`\"{}}\\*
{\ttfamily \#include \char`\"{}kolejka.\-hpp\char`\"{}}\\*
{\ttfamily \#include \char`\"{}plik.\-hpp\char`\"{}}\\*
Wykres zależności załączania dla main.\-cpp\-:
\subsection*{Funkcje}
\begin{DoxyCompactItemize}
\item 
int \hyperlink{main_8cpp_ae66f6b31b5ad750f1fe042a706a4e3d4}{main} ()
\end{DoxyCompactItemize}


\subsection{Dokumentacja funkcji}
\hypertarget{main_8cpp_ae66f6b31b5ad750f1fe042a706a4e3d4}{\index{main.\-cpp@{main.\-cpp}!main@{main}}
\index{main@{main}!main.cpp@{main.\-cpp}}
\subsubsection[{main}]{\setlength{\rightskip}{0pt plus 5cm}int main (
\begin{DoxyParamCaption}
{}
\end{DoxyParamCaption}
)}}\label{main_8cpp_ae66f6b31b5ad750f1fe042a706a4e3d4}


Definicja w linii 17 pliku main.\-cpp.



Oto graf wywołań dla tej funkcji\-:



\hypertarget{plik_8cpp}{\section{Dokumentacja pliku /home/karolina/\-Pulpit/\-L\-A\-B3nowe/prj/plik.cpp}
\label{plik_8cpp}\index{/home/karolina/\-Pulpit/\-L\-A\-B3nowe/prj/plik.\-cpp@{/home/karolina/\-Pulpit/\-L\-A\-B3nowe/prj/plik.\-cpp}}
}


Definicja funkcji Read.  


{\ttfamily \#include \char`\"{}plik.\-hpp\char`\"{}}\\*
Wykres zależności załączania dla plik.\-cpp\-:
\subsection*{Funkcje}
\begin{DoxyCompactItemize}
\item 
bool \hyperlink{plik_8cpp_a603d1d2e8cd3619abf4032fbf8cc212d}{Read} (const char $\ast$name, \hyperlink{class_stos}{Stos} $\ast$sto)
\item 
void \hyperlink{plik_8cpp_a27e694bd359044803046c9cec1f24141}{Write} (const char $\ast$name, \hyperlink{class_stos}{Stos} $\ast$sto)
\item 
bool \hyperlink{plik_8cpp_ac3bcb770ee00a4905d76cbe15e098718}{Read} (const char $\ast$name, \hyperlink{class_queue}{Queue} $\ast$que)
\item 
void \hyperlink{plik_8cpp_a0f170f6dccd0a4f11578296e435f6e54}{Write} (const char $\ast$name, \hyperlink{class_queue}{Queue} $\ast$que)
\end{DoxyCompactItemize}


\subsection{Opis szczegółowy}
Definicja funkcji Read. Definicja funkcji Write.

Funkcja odpowiedzialna za za czytanie Stosu.

Funkcja odpowiedzialna za zapisywanie do pliku Stosu

Funkcja odpowiedzialna za za czytanie Kolejki.

Funkcja odpowiedzialna za zapisywanie do pliku Kolejki 

Definicja w pliku \hyperlink{plik_8cpp_source}{plik.\-cpp}.



\subsection{Dokumentacja funkcji}
\hypertarget{plik_8cpp_a603d1d2e8cd3619abf4032fbf8cc212d}{\index{plik.\-cpp@{plik.\-cpp}!Read@{Read}}
\index{Read@{Read}!plik.cpp@{plik.\-cpp}}
\subsubsection[{Read}]{\setlength{\rightskip}{0pt plus 5cm}bool Read (
\begin{DoxyParamCaption}
\item[{const char $\ast$}]{name, }
\item[{{\bf Stos} $\ast$}]{sto}
\end{DoxyParamCaption}
)}}\label{plik_8cpp_a603d1d2e8cd3619abf4032fbf8cc212d}


Definicja w linii 8 pliku plik.\-cpp.



Oto graf wywołań dla tej funkcji\-:




Oto graf wywoływań tej funkcji\-:


\hypertarget{plik_8cpp_ac3bcb770ee00a4905d76cbe15e098718}{\index{plik.\-cpp@{plik.\-cpp}!Read@{Read}}
\index{Read@{Read}!plik.cpp@{plik.\-cpp}}
\subsubsection[{Read}]{\setlength{\rightskip}{0pt plus 5cm}bool Read (
\begin{DoxyParamCaption}
\item[{const char $\ast$}]{name, }
\item[{{\bf Queue} $\ast$}]{que}
\end{DoxyParamCaption}
)}}\label{plik_8cpp_ac3bcb770ee00a4905d76cbe15e098718}


Definicja w linii 50 pliku plik.\-cpp.



Oto graf wywołań dla tej funkcji\-:


\hypertarget{plik_8cpp_a27e694bd359044803046c9cec1f24141}{\index{plik.\-cpp@{plik.\-cpp}!Write@{Write}}
\index{Write@{Write}!plik.cpp@{plik.\-cpp}}
\subsubsection[{Write}]{\setlength{\rightskip}{0pt plus 5cm}void Write (
\begin{DoxyParamCaption}
\item[{const char $\ast$}]{name, }
\item[{{\bf Stos} $\ast$}]{sto}
\end{DoxyParamCaption}
)}}\label{plik_8cpp_a27e694bd359044803046c9cec1f24141}


Definicja w linii 35 pliku plik.\-cpp.



Oto graf wywołań dla tej funkcji\-:


\hypertarget{plik_8cpp_a0f170f6dccd0a4f11578296e435f6e54}{\index{plik.\-cpp@{plik.\-cpp}!Write@{Write}}
\index{Write@{Write}!plik.cpp@{plik.\-cpp}}
\subsubsection[{Write}]{\setlength{\rightskip}{0pt plus 5cm}void Write (
\begin{DoxyParamCaption}
\item[{const char $\ast$}]{name, }
\item[{{\bf Queue} $\ast$}]{que}
\end{DoxyParamCaption}
)}}\label{plik_8cpp_a0f170f6dccd0a4f11578296e435f6e54}


Definicja w linii 76 pliku plik.\-cpp.



Oto graf wywołań dla tej funkcji\-:



\hypertarget{plik_8hpp}{\section{Dokumentacja pliku /home/karolina/\-Pulpit/\-L\-A\-B3nowe/prj/plik.hpp}
\label{plik_8hpp}\index{/home/karolina/\-Pulpit/\-L\-A\-B3nowe/prj/plik.\-hpp@{/home/karolina/\-Pulpit/\-L\-A\-B3nowe/prj/plik.\-hpp}}
}


Definicja funkcji Read.  


{\ttfamily \#include $<$iostream$>$}\\*
{\ttfamily \#include $<$fstream$>$}\\*
{\ttfamily \#include \char`\"{}stos.\-hpp\char`\"{}}\\*
{\ttfamily \#include \char`\"{}kolejka.\-hpp\char`\"{}}\\*
Wykres zależności załączania dla plik.\-hpp\-:
Ten wykres pokazuje, które pliki bezpośrednio lub pośrednio załączają ten plik\-:
\subsection*{Funkcje}
\begin{DoxyCompactItemize}
\item 
bool \hyperlink{plik_8hpp_a603d1d2e8cd3619abf4032fbf8cc212d}{Read} (const char $\ast$name, \hyperlink{class_stos}{Stos} $\ast$sto)
\item 
void \hyperlink{plik_8hpp_a27e694bd359044803046c9cec1f24141}{Write} (const char $\ast$name, \hyperlink{class_stos}{Stos} $\ast$sto)
\item 
bool \hyperlink{plik_8hpp_ac3bcb770ee00a4905d76cbe15e098718}{Read} (const char $\ast$name, \hyperlink{class_queue}{Queue} $\ast$que)
\item 
void \hyperlink{plik_8hpp_a0f170f6dccd0a4f11578296e435f6e54}{Write} (const char $\ast$name, \hyperlink{class_queue}{Queue} $\ast$que)
\end{DoxyCompactItemize}


\subsection{Opis szczegółowy}
Definicja funkcji Read. Definicja funkcji Write.

Funkcja odpowiedzialna za za czytanie Stosu.

Funkcja odpowiedzialna za zapisywanie do pliku Stosu

Funkcja odpowiedzialna za za czytanie Kolejki.

Funkcja odpowiedzialna za zapisywanie do pliku Kolejki 

Definicja w pliku \hyperlink{plik_8hpp_source}{plik.\-hpp}.



\subsection{Dokumentacja funkcji}
\hypertarget{plik_8hpp_a603d1d2e8cd3619abf4032fbf8cc212d}{\index{plik.\-hpp@{plik.\-hpp}!Read@{Read}}
\index{Read@{Read}!plik.hpp@{plik.\-hpp}}
\subsubsection[{Read}]{\setlength{\rightskip}{0pt plus 5cm}bool Read (
\begin{DoxyParamCaption}
\item[{const char $\ast$}]{name, }
\item[{{\bf Stos} $\ast$}]{sto}
\end{DoxyParamCaption}
)}}\label{plik_8hpp_a603d1d2e8cd3619abf4032fbf8cc212d}


Definicja w linii 8 pliku plik.\-cpp.



Oto graf wywołań dla tej funkcji\-:




Oto graf wywoływań tej funkcji\-:


\hypertarget{plik_8hpp_ac3bcb770ee00a4905d76cbe15e098718}{\index{plik.\-hpp@{plik.\-hpp}!Read@{Read}}
\index{Read@{Read}!plik.hpp@{plik.\-hpp}}
\subsubsection[{Read}]{\setlength{\rightskip}{0pt plus 5cm}bool Read (
\begin{DoxyParamCaption}
\item[{const char $\ast$}]{name, }
\item[{{\bf Queue} $\ast$}]{que}
\end{DoxyParamCaption}
)}}\label{plik_8hpp_ac3bcb770ee00a4905d76cbe15e098718}


Definicja w linii 50 pliku plik.\-cpp.



Oto graf wywołań dla tej funkcji\-:


\hypertarget{plik_8hpp_a27e694bd359044803046c9cec1f24141}{\index{plik.\-hpp@{plik.\-hpp}!Write@{Write}}
\index{Write@{Write}!plik.hpp@{plik.\-hpp}}
\subsubsection[{Write}]{\setlength{\rightskip}{0pt plus 5cm}void Write (
\begin{DoxyParamCaption}
\item[{const char $\ast$}]{name, }
\item[{{\bf Stos} $\ast$}]{sto}
\end{DoxyParamCaption}
)}}\label{plik_8hpp_a27e694bd359044803046c9cec1f24141}


Definicja w linii 35 pliku plik.\-cpp.



Oto graf wywołań dla tej funkcji\-:


\hypertarget{plik_8hpp_a0f170f6dccd0a4f11578296e435f6e54}{\index{plik.\-hpp@{plik.\-hpp}!Write@{Write}}
\index{Write@{Write}!plik.hpp@{plik.\-hpp}}
\subsubsection[{Write}]{\setlength{\rightskip}{0pt plus 5cm}void Write (
\begin{DoxyParamCaption}
\item[{const char $\ast$}]{name, }
\item[{{\bf Queue} $\ast$}]{que}
\end{DoxyParamCaption}
)}}\label{plik_8hpp_a0f170f6dccd0a4f11578296e435f6e54}


Definicja w linii 76 pliku plik.\-cpp.



Oto graf wywołań dla tej funkcji\-:



\hypertarget{stos_8hpp}{\section{Dokumentacja pliku /home/karolina/\-Pulpit/\-L\-A\-B3nowe/prj/stos.hpp}
\label{stos_8hpp}\index{/home/karolina/\-Pulpit/\-L\-A\-B3nowe/prj/stos.\-hpp@{/home/karolina/\-Pulpit/\-L\-A\-B3nowe/prj/stos.\-hpp}}
}


Definicja klasy \hyperlink{class_stos}{Stos}.  


{\ttfamily \#include \char`\"{}tablica.\-hpp\char`\"{}}\\*
{\ttfamily \#include \char`\"{}lista.\-hpp\char`\"{}}\\*
Wykres zależności załączania dla stos.\-hpp\-:
Ten wykres pokazuje, które pliki bezpośrednio lub pośrednio załączają ten plik\-:
\subsection*{Komponenty}
\begin{DoxyCompactItemize}
\item 
class \hyperlink{class_stos}{Stos}
\end{DoxyCompactItemize}
\subsection*{Wyliczenia}
\begin{DoxyCompactItemize}
\item 
enum \hyperlink{stos_8hpp_ad7b974b79929c04ea204e3304ff8c776}{T\-Y\-P} \{ \hyperlink{stos_8hpp_ad7b974b79929c04ea204e3304ff8c776a1ad282f7a4661a467bdb03ba3a69fad6}{list}, 
\hyperlink{stos_8hpp_ad7b974b79929c04ea204e3304ff8c776a78331c231f0f1261205e7f04d31355fa}{stack}
 \}
\end{DoxyCompactItemize}


\subsection{Opis szczegółowy}
Definicja klasy \hyperlink{class_stos}{Stos}. Klasa odpowiadajaca za metody,, ktore wykonuja sie na stosie. Naleza do nich\-: push, pop, Isempty, Size, Show 

Definicja w pliku \hyperlink{stos_8hpp_source}{stos.\-hpp}.



\subsection{Dokumentacja typów wyliczanych}
\hypertarget{stos_8hpp_ad7b974b79929c04ea204e3304ff8c776}{\index{stos.\-hpp@{stos.\-hpp}!T\-Y\-P@{T\-Y\-P}}
\index{T\-Y\-P@{T\-Y\-P}!stos.hpp@{stos.\-hpp}}
\subsubsection[{T\-Y\-P}]{\setlength{\rightskip}{0pt plus 5cm}enum {\bf T\-Y\-P}}}\label{stos_8hpp_ad7b974b79929c04ea204e3304ff8c776}
\begin{Desc}
\item[Wartości wyliczeń]\par
\begin{description}
\index{list@{list}!stos.\-hpp@{stos.\-hpp}}\index{stos.\-hpp@{stos.\-hpp}!list@{list}}\item[{\em 
\hypertarget{stos_8hpp_ad7b974b79929c04ea204e3304ff8c776a1ad282f7a4661a467bdb03ba3a69fad6}{list}\label{stos_8hpp_ad7b974b79929c04ea204e3304ff8c776a1ad282f7a4661a467bdb03ba3a69fad6}
}]\index{stack@{stack}!stos.\-hpp@{stos.\-hpp}}\index{stos.\-hpp@{stos.\-hpp}!stack@{stack}}\item[{\em 
\hypertarget{stos_8hpp_ad7b974b79929c04ea204e3304ff8c776a78331c231f0f1261205e7f04d31355fa}{stack}\label{stos_8hpp_ad7b974b79929c04ea204e3304ff8c776a78331c231f0f1261205e7f04d31355fa}
}]\end{description}
\end{Desc}


Definicja w linii 7 pliku stos.\-hpp.


\hypertarget{tablica_8cpp}{\section{Dokumentacja pliku /home/karolina/\-Pulpit/\-L\-A\-B3nowe/prj/tablica.cpp}
\label{tablica_8cpp}\index{/home/karolina/\-Pulpit/\-L\-A\-B3nowe/prj/tablica.\-cpp@{/home/karolina/\-Pulpit/\-L\-A\-B3nowe/prj/tablica.\-cpp}}
}


Definicja konstruktora Pusch .  


{\ttfamily \#include \char`\"{}tablica.\-hpp\char`\"{}}\\*
Wykres zależności załączania dla tablica.\-cpp\-:


\subsection{Opis szczegółowy}
Definicja konstruktora Pusch . Definicja przeladowania operatora $\ast$ .

Definicja przeladowania operatora == .

Definicja przeladowania operatora = .

Definicja przeladowania operatora + .

Definicja konstruktora Show\-\_\-\-Tab .

Definicja konstruktora Isempty .

Definicja konstruktora Size .

Definicja konstruktora Pop\-\_\-\-Back .

Definicja konstruktora Pop .

Metoda dodaje elementy. Przy uzyciu capacity metoda zwraca ilo�� zaalokowanej pami�ci wyra�onej w liczbie znak�w.

Metoda kopiuje wszystkie elementy z tab do tymczasowego. Nastpenie odbywa sie czyszczenie tab, wymiana odbiektow miedzy vectorami, a potem rezerwuje miejsce w vectorze na drukrotnosc jego elementow.

Metoda kopiuje wszystkie elementy z tab do tymczasowego. Nastpenie odbywa sie czyszczenie tab, wymiana odbiektow miedzy vectorami, a potem rezerwuje miejsce w vectorze na drukrotnosc jego elementow..

W celu sprawdzania rozmiaru tablicy.

W celu sprawdzania czy tablica jest pusta.

Pokazuje tablice.

W celu lacznia dwoch tablic w jedna.

W celu przypisania dwoch tablic.

W celu lacznia porownania tablic.

W celu mnozenia tablic. 

Definicja w pliku \hyperlink{tablica_8cpp_source}{tablica.\-cpp}.


\hypertarget{tablica_8hpp}{\section{Dokumentacja pliku /home/karolina/\-Pulpit/\-L\-A\-B3nowe/prj/tablica.hpp}
\label{tablica_8hpp}\index{/home/karolina/\-Pulpit/\-L\-A\-B3nowe/prj/tablica.\-hpp@{/home/karolina/\-Pulpit/\-L\-A\-B3nowe/prj/tablica.\-hpp}}
}


Definicja klasy \hyperlink{class_tablica}{Tablica}.  


{\ttfamily \#include $<$vector$>$}\\*
{\ttfamily \#include $<$iostream$>$}\\*
Wykres zależności załączania dla tablica.\-hpp\-:
Ten wykres pokazuje, które pliki bezpośrednio lub pośrednio załączają ten plik\-:
\subsection*{Komponenty}
\begin{DoxyCompactItemize}
\item 
class \hyperlink{class_tablica}{Tablica}
\end{DoxyCompactItemize}


\subsection{Opis szczegółowy}
Definicja klasy \hyperlink{class_tablica}{Tablica}. Klasa odpowiadaj�ca za operacje wykonujace sie na tablicy takie jak\-: zamienienie elementow, odwrocenie tablicy, dodanie elementu, dodanie elementow, operator + operator, operato == 
\begin{DoxyParams}[1]{Parametry}
\mbox{\tt in}  & {\em tab} & -\/ kontener dynamicznej tablicy jednowymiarowej o typie double \\
\hline
\end{DoxyParams}


Definicja w pliku \hyperlink{tablica_8hpp_source}{tablica.\-hpp}.


\hypertarget{zegar_8cpp}{\section{Dokumentacja pliku /home/karolina/\-Pulpit/\-L\-A\-B3nowe/prj/zegar.cpp}
\label{zegar_8cpp}\index{/home/karolina/\-Pulpit/\-L\-A\-B3nowe/prj/zegar.\-cpp@{/home/karolina/\-Pulpit/\-L\-A\-B3nowe/prj/zegar.\-cpp}}
}


Definicja metody Start.  


{\ttfamily \#include \char`\"{}zegar.\-hpp\char`\"{}}\\*
Wykres zależności załączania dla zegar.\-cpp\-:


\subsection{Opis szczegółowy}
Definicja metody Start. Definicja metody Wynik.

Definicja metody Koniec.

Metoda, ktora powoduje start zegara.

Metoda, ktora powoduje stop zegara i oblicza czas.

Metoda, ktora wyswietla wynik. 

Definicja w pliku \hyperlink{zegar_8cpp_source}{zegar.\-cpp}.


\hypertarget{zegar_8hpp}{\section{Dokumentacja pliku /home/karolina/\-Pulpit/\-L\-A\-B3nowe/prj/zegar.hpp}
\label{zegar_8hpp}\index{/home/karolina/\-Pulpit/\-L\-A\-B3nowe/prj/zegar.\-hpp@{/home/karolina/\-Pulpit/\-L\-A\-B3nowe/prj/zegar.\-hpp}}
}


Definicja klasy \hyperlink{class_tablica}{Tablica}.  


{\ttfamily \#include $<$ctime$>$}\\*
{\ttfamily \#include $<$iostream$>$}\\*
Wykres zależności załączania dla zegar.\-hpp\-:
Ten wykres pokazuje, które pliki bezpośrednio lub pośrednio załączają ten plik\-:
\subsection*{Komponenty}
\begin{DoxyCompactItemize}
\item 
class \hyperlink{class_zegar}{Zegar}
\end{DoxyCompactItemize}


\subsection{Opis szczegółowy}
Definicja klasy \hyperlink{class_tablica}{Tablica}. Klasa odpowiadaj�ca za operacje wykonujace sie na tablicy takie jak\-: zamienienie elementow, odwrocenie tablicy, dodanie elementu, dodanie elementow, operator + operator, operato == 
\begin{DoxyParams}[1]{Parametry}
\mbox{\tt in}  & {\em clock\-\_\-t} & start, koniec -\/ zmienne przechowuja aktualny czas systemu \\
\hline
\mbox{\tt in}  & {\em czas} & -\/ przechowuje roznice czasow koniec i start \\
\hline
\end{DoxyParams}


Definicja w pliku \hyperlink{zegar_8hpp_source}{zegar.\-hpp}.


\addcontentsline{toc}{part}{Indeks}
\printindex
\end{document}
