\hypertarget{tablica_8cpp}{\section{Dokumentacja pliku /home/karolina/\-Pulpit/\-L\-A\-B3nowe/prj/tablica.cpp}
\label{tablica_8cpp}\index{/home/karolina/\-Pulpit/\-L\-A\-B3nowe/prj/tablica.\-cpp@{/home/karolina/\-Pulpit/\-L\-A\-B3nowe/prj/tablica.\-cpp}}
}


Definicja konstruktora Pusch .  


{\ttfamily \#include \char`\"{}tablica.\-hpp\char`\"{}}\\*
Wykres zależności załączania dla tablica.\-cpp\-:


\subsection{Opis szczegółowy}
Definicja konstruktora Pusch . Definicja przeladowania operatora $\ast$ .

Definicja przeladowania operatora == .

Definicja przeladowania operatora = .

Definicja przeladowania operatora + .

Definicja konstruktora Show\-\_\-\-Tab .

Definicja konstruktora Isempty .

Definicja konstruktora Size .

Definicja konstruktora Pop\-\_\-\-Back .

Definicja konstruktora Pop .

Metoda dodaje elementy. Przy uzyciu capacity metoda zwraca ilo�� zaalokowanej pami�ci wyra�onej w liczbie znak�w.

Metoda kopiuje wszystkie elementy z tab do tymczasowego. Nastpenie odbywa sie czyszczenie tab, wymiana odbiektow miedzy vectorami, a potem rezerwuje miejsce w vectorze na drukrotnosc jego elementow.

Metoda kopiuje wszystkie elementy z tab do tymczasowego. Nastpenie odbywa sie czyszczenie tab, wymiana odbiektow miedzy vectorami, a potem rezerwuje miejsce w vectorze na drukrotnosc jego elementow..

W celu sprawdzania rozmiaru tablicy.

W celu sprawdzania czy tablica jest pusta.

Pokazuje tablice.

W celu lacznia dwoch tablic w jedna.

W celu przypisania dwoch tablic.

W celu lacznia porownania tablic.

W celu mnozenia tablic. 

Definicja w pliku \hyperlink{tablica_8cpp_source}{tablica.\-cpp}.

